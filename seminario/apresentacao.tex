\documentclass[aspectratio=169]{beamer}
\usepackage[utf8]{inputenc}
\usepackage[T1]{fontenc}
\usepackage[portuguese]{babel}
\usepackage{amsmath,amsfonts,amssymb}
\usepackage{graphicx}
\usepackage{tikz}
\usepackage{booktabs}
\usepackage{xcolor}

% Tema e cores
\usetheme{Madrid}
\usecolortheme{default}
\definecolor{azulescuro}{HTML}{1f4e79}
\definecolor{cinza}{HTML}{5f5f5f}
\setbeamercolor{title}{fg=azulescuro}
\setbeamercolor{frametitle}{fg=azulescuro}
\setbeamercolor{structure}{fg=azulescuro}

% Informações do documento
\title{Learning Multiple Layers of Features from Tiny Images}
\subtitle{Alex Krizhevsky (2009)}
\author{Apresentação para Banca de Pós-Graduação}
\institute{University of Toronto \\ Orientador: Geoffrey Hinton}
\date{\today}

\begin{document}

% Slide de título
\begin{frame}
\titlepage
\end{frame}

% Sumário
\begin{frame}
\frametitle{Sumário}
\tableofcontents
\end{frame}

\section{Contexto Histórico}

\begin{frame}
\frametitle{Deep Learning em 2009 - O Renascimento}
\begin{columns}
\begin{column}{0.5\textwidth}
\textbf{Breakthrough Recente:}
\begin{itemize}
    \item \textcolor{azulescuro}{\textbf{2006}}: Hinton introduz DBNs
    \item \textcolor{azulescuro}{\textbf{Problema}}: Como treinar redes profundas?
    \item \textcolor{azulescuro}{\textbf{Solução}}: Pré-treinamento não supervisionado
\end{itemize}
\end{column}
\begin{column}{0.5\textwidth}
\textbf{Desafios da Época:}
\begin{itemize}
    \item ❌ Vanishing gradient problem
    \item ❌ Falta de grandes datasets
    \item ❌ Limitações computacionais
    \item ❌ Feature engineering manual
\end{itemize}
\end{column}
\end{columns}
\end{frame}

\section{Motivação}

\begin{frame}
\frametitle{Motivação do Trabalho}
\begin{alertblock}{Falhas Anteriores}
\begin{itemize}
    \item MIT/NYU falharam com "80 million tiny images"
    \item Modelos aprendiam apenas filtros ruidosos
\end{itemize}
\end{alertblock}

\textbf{Objetivos Principais:}
\begin{enumerate}
    \item \textcolor{azulescuro}{\textbf{Modelagem generativa}} eficaz de imagens
    \item \textcolor{azulescuro}{\textbf{Datasets confiáveis}} para benchmarking  
    \item \textcolor{azulescuro}{\textbf{Paralelização}} para escalabilidade
\end{enumerate}
\end{frame}

\section{Fundamentos Teóricos}

\begin{frame}
\frametitle{ZCA Whitening - Fundamento Teórico}
\textbf{Por que Whitening?}
\begin{itemize}
    \item Remove correlações de \textcolor{azulescuro}{\textbf{segunda ordem}}
    \item Força foco em correlações de \textcolor{azulescuro}{\textbf{alta ordem}}
    \item \textcolor{azulescuro}{\textbf{Fundamental}} para sucesso do método
\end{itemize}

\begin{exampleblock}{Formulação Matemática}
\begin{align}
C &= \frac{1}{n-1} XX^T \\
W &= \frac{1}{\sqrt{n-1}} P D^{-1/2} P^T \\
Y &= WX
\end{align}
\end{exampleblock}
\end{frame}

\begin{frame}
\frametitle{Restricted Boltzmann Machines}
\begin{columns}
\begin{column}{0.5\textwidth}
\begin{center}
\textbf{[IMAGEM: Arquitetura RBM]}\\
\textit{Unidades visíveis e ocultas\\sem conexões intra-camada}
\end{center}
\end{column}
\begin{column}{0.5\textwidth}
\begin{exampleblock}{Função de Energia (RBM Binária)}
\begin{equation}
E(v,h) = -\sum_{i,j} v_i h_j w_{ij} - \sum_i v_i b_i^v - \sum_j h_j b_j^h
\end{equation}
\textbf{Onde:}
\begin{itemize}
    \item $v$: estado das unidades visíveis
    \item $h$: estado das unidades ocultas
    \item $w_{ij}$: pesos entre unidades
\end{itemize}
\end{exampleblock}
\end{column}
\end{columns}
\end{frame}

\begin{frame}
\frametitle{RBM Gaussiana-Bernoulli}
\textbf{Para Dados Reais (Intensidades de Pixels):}

\begin{exampleblock}{Função de Energia Gaussiana}
\begin{equation}
E(v,h) = \sum_{i=1}^V \frac{(v_i - b_i^v)^2}{2\sigma_i^2} - \sum_{j=1}^H b_j^h h_j - \sum_{i=1}^V \sum_{j=1}^H \frac{v_i}{\sigma_i} h_j w_{ij}
\end{equation}
\end{exampleblock}

\textbf{Distribuições Condicionais:}
\begin{itemize}
    \item \textcolor{azulescuro}{\textbf{Visíveis}}: Gaussianas com média dependente de $h$
    \item \textcolor{azulescuro}{\textbf{Ocultas}}: Bernoulli com probabilidade sigmoid
\end{itemize}
\end{frame}

\begin{frame}
\frametitle{Contrastive Divergence (CD-1)}
\begin{center}
\textbf{[IMAGEM: Procedimento CD-N]}\\
\textit{Sampling alternado para estimação eficiente}
\end{center}

\begin{exampleblock}{Atualização de Pesos}
\begin{equation}
\Delta w_{ij} = \epsilon \left( \mathbb{E}_{data}[v_i h_j] - \mathbb{E}_{model}[v_i h_j] \right)
\end{equation}
\begin{itemize}
    \item $\mathbb{E}_{data}$: Expectativa com visíveis fixadas
    \item $\mathbb{E}_{model}$: Expectativa aproximada por CD
\end{itemize}
\end{exampleblock}
\end{frame}

\section{Desenvolvimento Prático}

\begin{frame}
\frametitle{Problema Inicial}
\begin{center}
\textbf{[IMAGEM: Filtros Sem Sentido]}\\
\textit{Filtros ruidosos aprendidos em dados whitened}
\end{center}

\begin{alertblock}{Causa do Problema}
\begin{itemize}
    \item \textcolor{azulescuro}{\textbf{Ruído de alta frequência}} dominante
    \item \textcolor{azulescuro}{\textbf{Correlações complexas}} não capturadas  
    \item \textcolor{azulescuro}{\textbf{Necessidade}}: Estratégias mais sofisticadas
\end{itemize}
\end{alertblock}
\end{frame}

\begin{frame}
\frametitle{Solução: Estratégia de Patches}
\begin{center}
\textbf{[IMAGEM: Segmentação em Patches]}\\
\textit{Divisão 32×32 em 25 patches de 8×8}
\end{center}

\textbf{Abordagem:}
\begin{itemize}
    \item \textcolor{azulescuro}{\textbf{25 patches}} de 8×8 pixels
    \item \textcolor{azulescuro}{\textbf{1 patch global}} subsampled
    \item \textcolor{azulescuro}{\textbf{26 RBMs independentes}}
    \item \textcolor{azulescuro}{\textbf{Redução}}: Complexidade dimensional
\end{itemize}
\end{frame}

\begin{frame}
\frametitle{Breakthrough: Filtros de Qualidade}
\begin{center}
\textbf{[IMAGEM: Filtros 8×8 de Qualidade]}\\
\textit{Detectores de borda aprendidos com sucesso}
\end{center}

\begin{exampleblock}{Características dos Filtros}
\begin{itemize}
    \item \textcolor{azulescuro}{\textbf{Filtros coloridos}}: Baixa frequência
    \item \textcolor{azulescuro}{\textbf{Filtros P\&B}}: Alta frequência
    \item \textcolor{azulescuro}{\textbf{Interpretação}}: Posição precisa + cor aproximada
\end{itemize}
\end{exampleblock}
\end{frame}

\section{Resultados}

\begin{frame}
\frametitle{Resultados de Classificação - CIFAR-10}
\begin{table}
\centering
\begin{tabular}{lc}
\toprule
\textbf{Método} & \textbf{Erro (\%)} \\
\midrule
Logistic Regression (raw pixels) & $\sim$40 \\
Logistic Regression (whitened pixels) & $\sim$37 \\
Logistic Regression (RBM features) & \textcolor{azulescuro}{\textbf{$\sim$22}} \\
\textbf{Neural Network (RBM init)} & \textcolor{azulescuro}{\textbf{$\sim$18.5}} \\
\bottomrule
\end{tabular}
\end{table}

\begin{alertblock}{Insights Principais}
\begin{itemize}
    \item \textcolor{azulescuro}{\textbf{Features RBM}} $\gg$ pixels crus
    \item \textcolor{azulescuro}{\textbf{Dados não-whitened}} melhores para RBMs
\end{itemize}
\end{alertblock}
\end{frame}

\begin{frame}
\frametitle{Dataset CIFAR - Contribuição Duradoura}
\begin{columns}
\begin{column}{0.5\textwidth}
\textbf{CIFAR-10:}
\begin{itemize}
    \item \textcolor{azulescuro}{\textbf{10 classes}}: airplane, automobile, bird, cat, deer, dog, frog, horse, ship, truck
    \item \textcolor{azulescuro}{\textbf{60.000 imagens}}
    \item \textcolor{azulescuro}{\textbf{50k treino + 10k teste}}
\end{itemize}
\end{column}
\begin{column}{0.5\textwidth}
\textbf{CIFAR-100:}
\begin{itemize}
    \item \textcolor{azulescuro}{\textbf{100 classes}}
    \item \textcolor{azulescuro}{\textbf{20 superclasses}}
    \item \textcolor{azulescuro}{\textbf{600 imagens/classe}}
\end{itemize}
\end{column}
\end{columns}

\begin{exampleblock}{Impacto}
\begin{itemize}
    \item \textcolor{azulescuro}{\textbf{Benchmark fundamental}} até hoje
    \item \textcolor{azulescuro}{\textbf{Base para pesquisas}} em computer vision
    \item \textcolor{azulescuro}{\textbf{Metodologia rigorosa}} de rotulação
\end{itemize}
\end{exampleblock}
\end{frame}

\section{Inovação Técnica}

\begin{frame}
\frametitle{Paralelização Eficiente}
\textbf{Desafio Computacional:}
\begin{itemize}
    \item \textcolor{azulescuro}{\textbf{8000 visíveis × 20000 ocultas}}
    \item \textcolor{azulescuro}{\textbf{Milhões de imagens}}
    \item \textcolor{azulescuro}{\textbf{Necessidade}}: Distribuição eficiente
\end{itemize}

\textbf{Algoritmo Desenvolvido:}
\begin{itemize}
    \item \textcolor{azulescuro}{\textbf{Divisão por máquinas}}: Subset de unidades
    \item \textcolor{azulescuro}{\textbf{Sincronização}}: Após cada sampling
    \item \textcolor{azulescuro}{\textbf{Comunicação mínima}}: Apenas bits
\end{itemize}

\begin{exampleblock}{Custo de Comunicação}
\begin{equation}
\text{Total} = 48 \times (K-1) \text{ MB por batch}
\end{equation}
\end{exampleblock}
\end{frame}

\begin{frame}
\frametitle{Resultados de Paralelização}
\begin{center}
\textbf{[IMAGEM: Gráficos de Speedup]}\\
\textit{Escalabilidade excelente até 8 máquinas}
\end{center}

\textbf{Resultados:}
\begin{itemize}
    \item \textcolor{azulescuro}{\textbf{Speedup quase linear}} até 8 máquinas
    \item \textcolor{azulescuro}{\textbf{Comunicação negligível}} para dados binários
    \item \textcolor{azulescuro}{\textbf{Double precision}}: Melhor escalabilidade
    \item \textcolor{azulescuro}{\textbf{Eficiência mantida}} em todas configurações
\end{itemize}
\end{frame}

\section{Legado e Impacto}

\begin{frame}
\frametitle{Impacto Científico}
\begin{columns}
\begin{column}{0.5\textwidth}
\textbf{Contribuições Imediatas:}
\begin{itemize}
    \item \textcolor{azulescuro}{\textbf{Prova de conceito}}: Deep learning para imagens reais
    \item \textcolor{azulescuro}{\textbf{Benchmarks duradouros}}: CIFAR ainda usado
    \item \textcolor{azulescuro}{\textbf{Paralelização pioneira}}
    \item \textcolor{azulescuro}{\textbf{Metodologia sólida}}
\end{itemize}
\end{column}
\begin{column}{0.5\textwidth}
\textbf{Influência Futura:}
\begin{itemize}
    \item \textcolor{azulescuro}{\textbf{AlexNet (2012)}}: Revolução de Krizhevsky
    \item \textcolor{azulescuro}{\textbf{Frameworks modernos}}: Horovod, Ray
    \item \textcolor{azulescuro}{\textbf{Preprocessing}}: ZCA ainda relevante
\end{itemize}
\end{column}
\end{columns}
\end{frame}

\begin{frame}
\frametitle{Conexões com Desenvolvimentos Atuais}
\textbf{Energy-Based Models (2020+):}
\begin{itemize}
    \item \textcolor{azulescuro}{\textbf{EBGAN}}: RBM + GANs
    \item \textcolor{azulescuro}{\textbf{JEM}}: Classificação + geração unificada
    \item \textcolor{azulescuro}{\textbf{Score-based}}: Gradientes de energia
\end{itemize}

\textbf{Contrastive Learning:}
\begin{itemize}
    \item \textcolor{azulescuro}{\textbf{SimCLR, MoCo}}: Evolução de CD
    \item \textcolor{azulescuro}{\textbf{CLIP}}: Contrastivo multimodal
    \item \textcolor{azulescuro}{\textbf{Self-supervised}}: Princípios de CD
\end{itemize}

\textbf{Arquiteturas Modernas:}
\begin{itemize}
    \item \textcolor{azulescuro}{\textbf{Vision Transformers}}: Patches similares
    \item \textcolor{azulescuro}{\textbf{ResNets}}: Skip connections das DBNs
\end{itemize}
\end{frame}

\section{Conclusões}

\begin{frame}
\frametitle{Conclusões}
\begin{alertblock}{Trabalho Revolucionário}
Este trabalho estabeleceu as \textcolor{azulescuro}{\textbf{bases da revolução}} do deep learning
\end{alertblock}

\textbf{Legado Duradouro:}
\begin{itemize}
    \item \textcolor{azulescuro}{\textbf{Ponte histórica}}: Hinton (2006) $\rightarrow$ AlexNet (2012)
    \item \textcolor{azulescuro}{\textbf{Infraestrutura}}: Datasets e algoritmos para comunidade
    \item \textcolor{azulescuro}{\textbf{Metodologia}}: Padrões de avaliação e visualização
    \item \textcolor{azulescuro}{\textbf{Escalabilidade}}: Computação distribuída
\end{itemize}

\begin{exampleblock}{Importância}
Demonstrou \textcolor{azulescuro}{\textbf{viabilidade prática}} do deep learning em dados reais
\end{exampleblock}
\end{frame}

\begin{frame}
\frametitle{Perguntas?}
\begin{center}
{\Large \textcolor{azulescuro}{\textbf{Obrigado pela atenção!}}}

\vspace{1cm}

\textbf{Tópicos para Discussão:}
\begin{itemize}
    \item Comparação com métodos atuais
    \item Aplicações modernas de RBMs
    \item Evolução para Transformers
    \item Paralelização em deep learning atual
\end{itemize}

\vspace{0.5cm}

\textbf{Material Disponível:}\\
Resumo completo $\bullet$ Código exemplo $\bullet$ Bibliografia
\end{center}
\end{frame}

\end{document}
